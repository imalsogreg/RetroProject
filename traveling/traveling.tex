% Template for PLoS
% Version 1.0 January 2009
%
% To compile to pdf, run:
% latex plos.template
% bibtex plos.template
% latex plos.template
% latex plos.template
% dvipdf plos.template

\documentclass[10pt]{article}

% amsmath package, useful for mathematical formulas
\usepackage{amsmath}
% amssymb package, useful for mathematical symbols
\usepackage{amssymb}

% graphicx package, useful for including eps and pdf graphics
% include graphics with the command \includegraphics
\usepackage{graphicx}

% cite package, to clean up citations in the main text. Do not remove.
\usepackage{cite}

\usepackage{color} 

% Use doublespacing - comment out for single spacing
%\usepackage{setspace} 
%\doublespacing


% Text layout
\topmargin 0.0cm
\oddsidemargin 0.5cm
\evensidemargin 0.5cm
\textwidth 16cm 
\textheight 21cm

% Bold the 'Figure #' in the caption and separate it with a period
% Captions will be left justified
\usepackage[labelfont=bf,labelsep=period,justification=raggedright]{caption}

% Use the PLoS provided bibtex style
\bibliographystyle{plos2009}

% Remove brackets from numbering in List of References
\makeatletter
\renewcommand{\@biblabel}[1]{\quad#1.}
\makeatother


% Leave date blank
\date{}

\pagestyle{myheadings}
%% ** EDIT HERE **


%% ** EDIT HERE **
%% PLEASE INCLUDE ALL MACROS BELOW

%% END MACROS SECTION

\begin{document}

% Title must be 150 characters or less
\begin{flushleft}
{\Large
\textbf{Spatial Information Synchrony in a Desynchronized Circuit}
}
% Insert Author names, affiliations and corresponding author email.
\\
Gregory J. Hale$^{1}$, 
Hector Penagos$^{2}$, 
Matthew A. Wilson$^{3,\ast}$
\\
\bf{1} Gregory Hale Brain and Cognitive Sciences, Massachusetts Institute of Technology, Cambridge, Massachusetts, USA
\\
\bf{2} Hector Penagos Brain and Cognitive Sciences, Massachusetts Institute of Technology, Cambridge, Massachusetts, USA
\\
\bf{3} Matthew Wilson Brain and Cognitive Sciences, Massachusetts Institute of Technology, Cambridge, Massachusetts, USA
\\
$\ast$ E-mail: Corresponding imalsogreg@gmail.com
\end{flushleft}

% Please keep the abstract between 250 and 300 words
\section*{Abstract}
Brain areas involved in mnemonic and spatial processing are locked to the common underlying theta (8-12 Hz) oscillation.  However, the various subregions of hippocampus, the different layers of entorhinal cortex, and subcortical areas are each maximally activated during different theta phases.  And within hippocampal layer CA1, theta phase preferences are offset in a gradient manner.  Theta not only paces neural excitability; it influences spatial information encoding by organizing the timing of place cell ensembles into temporally precise sequences.  Using multi-site tetrode arrays, we sought to determine the impact of theta timing offsets on the unity of spatial representations.  We show that spatial information content within CA1 is synchronized, and that CA3 information content is synchronized with that in CA1, despite theta phase offsets that would predict timing offsets of up to 30 ms.  Rather that desynchronizing representations, these theta offsets modulate the phase of sequence encoding at which the respective areas are most active.  This finding is at odds with prior models strictly linking theta phase to place cell spike timing. Adding a baseline excitatory drive to each area according to that area's phase offset brings the population information into synchrony and accounts for anatomical gradations in spatial receptive field shape.  These results show that fine-timescale information coding can be decoupled from underlying differences in timing of excitatory drive.

% Please keep the Author Summary between 150 and 200 words
% Use first person. PLoS ONE authors please skip this step. 
% Author Summary not valid for PLoS ONE submissions.   
\section*{Author Summary}
Theta oscillations have a local role assembling spatial information in individual brain structures, and a global role in coordinating activity between those structures.  But theta oscillations aren't synchronized across brain structures - different structures phase lock to different phases, and within the hippocampal region CA1, there is a continuous gradient of phase offsets.  Litte is known about how information movees from area to area in the hippocampus, but precise theta timing is known to have an important role.  Therefore a first step in understanding information flow is to assess whether these timing offsets in peak excitation are manifest as timing offsets in information representation.  By recording ensembles of neurons simultaneously throughout hippocampal CA1 and in CA3, we directly measured the level of excitation synchrony and information content synchrony.  We found that fine-timescale information content is measurably less than that of the offsets in theta, and not measurably different from 0.  Modifying a simple model of the interaction between neural encoding and theta oscillations by supplying each brain area with a baseline excitation change that cancels out the predicted effect of theta time offsets brings expecting information content back into synchrony (by definition), while accounting for previously documented differences in the tuning properties of individual neurons.  Information content is decoupled from offsets in underlying excitation in a way that preserves information synchrony, while allowing different brain areas to be preferrentially active during different phases of encoding.

\section*{Introduction}
Rhythmic modulation of neural firing is ubiquitous in the brain structures that underly mnemonic and spatial processes: the medial entorhinal cortex, the subiculum, anterior thalamic nuclei, retrosplenial cortex, and the various layers of the hippocampus.  Theta (8-12 Hz) oscillations have play an important role at two levels. On the global level they coordinate activity between connected brain regions \cite{Siapas2006, Jones2006, Serota2010, Colgin2011}. Locally they shape the fine-timescale properties of information coding within brain areas \cite{Recce1993,Skaggs1996,Mehta2002,Leutgeb2010}.

Unifying these two roles for theta oscillations is difficult, because they make conflicting demands on the details of how neurons interact with the oscillation.  Mizuseki et. al. \cite{Mizuseki2009} point out that time offsets theta oscillations across brain regions fail to match the sequence  of time offsets predicted by monosynaptic delays between connected areas.  For example, pikes should only take 10ms TODO_CHECK to move from entorhinal cortext to the dentage gyrus, but the peak activation times of these two areas differs by about 97 ms (over half of a theta cycle).  This phenomenon is acceptable to the global account of theta; it allows for the opening of 'temporal windows' of processing between sequential anatomical processing stages.  But it is at odds with intuitive and formal models of the fine timescale spiking of place cells TODO_CITE.  Place cells in dentate gyrus are generated from the summed spiking activity of their inputs, and are expected to follow the timing of their inputs closely TODO_CITE.

Colgin et. al. present data in support of a model associating particular phases of the theta oscillations of CA1 with the opening of specific communication channels to either CA3 or the entorhinal cortex \cite{Colgin2009}.  The tension between theta's local and global roles is apparent here, too.  Local effects on 

\section*{Results}

\subsection*{Subsection 1}

\subsection*{Subsection 2}

\section*{Discussion}
\cite{Mizuseki 2012}


% You may title this section "Methods" or "Models". 
% "Models" is not a valid title for PLoS ONE authors. However, PLoS ONE
% authors may use "Analysis" 
\section*{Materials and Methods}

% Do NOT remove this, even if you are not including acknowledgments
\section*{Acknowledgments}


%\section*{References}
% The bibtex filename
\bibliography{template}

\section*{Figure Legends}
%\begin{figure}[!ht]
%\begin{center}
%%\includegraphics[width=4in]{figure_name.2.eps}
%\end{center}
%\caption{
%{\bf Bold the first sentence.}  Rest of figure 2  caption.  Caption 
%should be left justified, as specified by the options to the caption 
%package.
%}
%\label{Figure_label}
%\end{figure}


\section*{Tables}
%\begin{table}[!ht]
%\caption{
%\bf{Table title}}
%\begin{tabular}{|c|c|c|}
%table information
%\end{tabular}
%\begin{flushleft}Table caption
%\end{flushleft}
%\label{tab:label}
% \end{table}

\end{document}

