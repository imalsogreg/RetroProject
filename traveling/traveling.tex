% Template for PLoS
% Version 1.0 January 2009
%
% To compile to pdf, run:
% latex plos.template
% bibtex plos.template
% latex plos.template
% latex plos.template
% dvipdf plos.template

\documentclass[10pt]{article}

% amsmath package, useful for mathematical formulas
\usepackage{amsmath}
% amssymb package, useful for mathematical symbols
\usepackage{amssymb}

% graphicx package, useful for including eps and pdf graphics
% include graphics with the command \includegraphics
\usepackage{graphicx}

% cite package, to clean up citations in the main text. Do not remove.
\usepackage{cite}

\usepackage{color} 

% Use doublespacing - comment out for single spacing
%\usepackage{setspace} 
%\doublespacing


% Text layout
\topmargin 0.0cm
\oddsidemargin 0.5cm
\evensidemargin 0.5cm
\textwidth 16cm 
\textheight 21cm

% Bold the 'Figure #' in the caption and separate it with a period
% Captions will be left justified
\usepackage[labelfont=bf,labelsep=period,justification=raggedright]{caption}

% Use the PLoS provided bibtex style
\bibliographystyle{plos2009}

% Remove brackets from numbering in List of References
\makeatletter
\renewcommand{\@biblabel}[1]{\quad#1.}
\makeatother


% Leave date blank
\date{}

\pagestyle{myheadings}
%% ** EDIT HERE **


%% ** EDIT HERE **
%% PLEASE INCLUDE ALL MACROS BELOW

%% END MACROS SECTION

\begin{document}

% Title must be 150 characters or less
\begin{flushleft}
{\Large
\textbf{Spatial Information Synchrony in a Desynchronized Circuit}
}
% Insert Author names, affiliations and corresponding author email.
\\
Gregory J. Hale$^{1}$, 
Hector Penagos$^{2}$, 
Matthew A. Wilson$^{3,\ast}$
\\
\bf{1} Gregory Hale Brain and Cognitive Sciences, Massachusetts Institute of Technology, Cambridge, Massachusetts, USA
\\
\bf{2} Hector Penagos Brain and Cognitive Sciences, Massachusetts Institute of Technology, Cambridge, Massachusetts, USA
\\
\bf{3} Matthew Wilson Brain and Cognitive Sciences, Massachusetts Institute of Technology, Cambridge, Massachusetts, USA
\\
$\ast$ E-mail: Corresponding imalsogreg@gmail.com
\end{flushleft}

% Please keep the abstract between 250 and 300 words
\section*{Abstract}
Brain areas involved in mnemonic and spatial processing are locked to an underlying 8-12 Hz oscillation known as the theta rhythm.  Different layers of entorhinal cortex, subcortical areas, and hippocampal subfields are each maximally activated during different theta phases; even within field CA1, theta-locked excitation is offset in a gradient manner.  In addition to pacing cell excitability, theta influences spatial information processing by organizing the timing of place cell ensembles into temporally precise sequences. We sought to determine the impact of theta timing offsets on the coordination of spatial representations in ensembles of place cells in different brain areas.

Along the CA1 septal-temporal axis, and between CA1 and CA3, we found spatial information content to be synchronized to within 5 ms, despite a time offset in theta on the order of 30 ms. Offsets in excitability manifest as a subtle tendency for different brain areas to be more active at different phases of the theta cycle, but this is independent of the encoded spatial information. The same degree of inter-area synchrony is apparent in the hippocampus of stationary rats, which sporadically replays sequences of spatial locations.

This observed information synchrony in the context of desynchronized excitation provides a constraint for future models of fast-timescale space coding. For instance, our prior excitation-to-phase model predicts desynchronized spatial information, but bringing it into alignment with our new data through compensatory inhibition preduces a pattern of single-cell spatial tuning that has 
Integrating these findings into a model of theta phase encoding can accoun for previous observations of diverse It also has implications for the integration of spatial signals in downstream structures, which may be befuddled by anything short of a coherent message from converging inputs.

This finding is at odds with prior models that strictly link theta phase to place cell spike timing. Adding a baseline excitatory drive to each area according to that area's phase offset brings the population information into synchrony and accounts for anatomical gradations in spatial receptive field shape.  These results show that fine-timescale information coding can be decoupled from underlying differences in timing of excitatory drive.

% Please keep the Author Summary between 150 and 200 words
% Use first person. PLoS ONE authors please skip this step. 
% Author Summary not valid for PLoS ONE submissions.   
\section*{Author Summary}
The hippocampal code for space is fundamentally based on time-compressed sequences of coordinated population spiking. The spikes of place cells (hippocampal neurons that spike only in limited regions of space) are organized into compressed sequences, with neurons tuned for an animal's immediate position firing spikes that are immediately followed by spikes from cells tuned to locations just ahead of the rat (CITATIONS). Treating the mean tuning preference of the neurons firing at a given instant as the position represented by the hippocampus at that instant, the representations 'moves' with great coordination, at approximately 8 times the physical speed of rat, to a distance about 1 meter ahead of the rat, then immediately jumps back to the rat's position, and quickly sweeps forward again. When rats stop to consume rewards on a track, the population spiking generally decreases, but is sporadically punctuated by sequences of ensemble spiking that 'move' from one positoin on the track to another, at about 8 times the rat's normal running speed, and now for observed distances up to 10 meters. 


Theta oscillations have a local role assembling spatial information in individual brain structures, and a global role in coordinating activity between those structures.  But theta oscillations aren't synchronized across brain structures - different structures phase lock to different phases, and within the hippocampal region CA1, there is a continuous gradient of phase offsets.  Litte is known about how information movees from area to area in the hippocampus, but precise theta timing is known to have an important role.  Therefore a first step in understanding information flow is to assess whether these timing offsets in peak excitation are manifest as timing offsets in information representation.  By recording ensembles of neurons simultaneously throughout hippocampal CA1 and in CA3, we directly measured the level of excitation synchrony and information content synchrony.  We found that fine-timescale information content is measurably less than that of the offsets in theta, and not measurably different from 0.  Modifying a simple model of the interaction between neural encoding and theta oscillations by supplying each brain area with a baseline excitation change that cancels out the predicted effect of theta time offsets brings expecting information content back into synchrony (by definition), while accounting for previously documented differences in the tuning properties of individual neurons.  Information content is decoupled from offsets in underlying excitation in a way that preserves information synchrony, while allowing different brain areas to be preferrentially active during different phases of encoding.

\section*{Introduction}
In many brain areas associated with spatial coding and episodic memory, neural activity is modulated by an 7-12 Hz oscillation called the theta rhythm \cite{vanderwolf1969hippocampal, buzsaki2002theta}.  The influence of theta on spatial and mnemonic information processing has been appreciated at two levels. On the global level, theta is thouht to coordinate activity between connected brain regions \cite{Siapas2006, Jones2006, Serota2010, Colgin2011}. Locally, theta shapes the fine-timescale properties of information coding within brain areas, by way of theta phase precession \cite{Recce1993,Skaggs1996,Mehta2002,Dragoi2006, Leutgeb2010}.

These two roles for theta oscillations are difficult to unify, because they make conflicting demands on the details of how neurons interact with the oscillation.  Mizuseki et. al. \cite{Mizuseki2009} point out that time offsets theta oscillations across brain regions fail to match the sequence  of time offsets predicted by monosynaptic delays between connected areas.  For example, pikes should only take 10ms (TODO CHECK) to move from entorhinal cortext to the dentage gyrus, but the peak activation times of these two areas differs by about 97 ms (over half of a theta cycle).  This phenomenon is acceptable to the global account of theta; it allows for the opening of 'temporal windows' of processing between sequential anatomical processing stages.  But it is at odds with intuitive and formal models of the fine timescale spiking of place cells (TODO CITE).  Place cells in dentate gyrus are generated from the summed spiking activity of their inputs, and are expected to follow the timing of their inputs closely (TODO CITE).

Colgin et. al. present data in support of a model associating particular phases of the theta oscillations of CA1 with the opening of specific communication channels to either CA3 or the entorhinal cortex \cite{Colgin2009}.  The tension between theta's local and global roles is apparent here, as well. To the extent that CA3-CA1 and entorhinal-CA1 communication is limited to narrow windows of theta phase. Contrary to this, place coding in CA1 involves a smooth transition through cell ensembles that extends over much of the theta cycle \cite{Foster, Gupta TODO}.

Lubenov and Siapas \cite{TODO} presented a novel finding and a supporting model that suggest a unification of global and local roles for the theta rhythm. Using large grids of tetrodes carefully positioned a uniform distance from the hippocampal cell layer, and sampling a large extent of the length of the hippocampus, they showed that the theta rhythm is not synchronous within hippocampal CA1. Instead, theta at the septal pole of CA1 are advanced in phase, theta in more posterior parts of CA1 are phase delayed, and theta measured inbetween has a graded delay. The combined activity of these delays roughly resembles a traveling wave with a peak of excitation that 'moves' down the hippocampal long axis once for every cycle of theta. The characteristics of this wave vary from cycle to cycle, but tend to have a spatial wavelength of 10mm (TODO check) and a preferred direction 45 degrees off of the septo-temporal axis.

If theta phase precession conforms to the anatomicaly sweeping of peak excitation, then theta sequences composed of sets of cells from different regions of CA1 would be similarly offset in time. The periodic replay of spatial sequences would begin slightly earlier in septal CA1 ensembles, and ensembles near intermediate CA1 would begin the same sequence about 30ms later (TODO check). Though this time shifting may seem to complicate attempts to square theta sequences with anatomical communication. However, it leads to an interesting prediction: that local regions of hippocampus begins a representation trajectory at offset times. Because of this, a downstream structure observing a snapshot of the spiking activity across the whole hippocampus would see different parts of the track encoded at different anatomical locations.

Alternatively, place cells may not conform to the timing offsets suggested by the traveling theta wave, and the encoded information may be temporally synchronized over large anatomical distances, despite the presumed timing differences in their underlying drive.

We set out to measure the timing relationship between theta waves and place cell sequences in order to begin to unify anatomical and population-coding accounts of information timing. We characterized the impact of spatial tuning and anatomical distance on the cofiring of pairs of place cells, as well as the timing relationships of population-encoded trajectories recovered from anatomically distinct groups of cells, both across CA1 and between CA1 and CA3. We found that in most cases, timing offsets in theta sequences were significantly more synchronized than the temporally offset excitatory waves that modulate them. We suggest that information synchrony may be decoupled from the mechanisms that modulate excitation. This decoupling could be achived in a trivial way, by stipulating that phase precession begins and ends according to an underlying source that is in fact synchronized across hippocampus; or it could be achieved through an active mechanism that supplies extra excitation to the regions that would otherwise be temporally delayed by the traveling theta wave.

\section*{Results}

\subsection*{Characteristics of the traveling theta wave}
We first characterized the timing of the theta rhythm, in a mannar essentially similar to that of Lubenov and Siapas \{cite TODO}, and found similar results. The peaks of local field potential occur at earlier times in tetrodes closer to the septal (most anterior) pole of CA1, and at later times in tetrodes closer to the medial third of CA1. Every millimeter of movement along a line of best fit produces approximately 15 +/- 5ms (TODO) of LFP delay. The best-fit line of phase delay tended not to lay perfectly along the septo-temporal axis, but instead cut obliquely across CA1. 

Using multiunit firing rate (MUA) instead of LFP to estimate the underlying rhythm produced qualitatively similar results, with some quantitative differences. The goodness-of-fit across many tetrodes to a single traveling wave model was lower for MUA than LFP (r^2^ = 0.9 +/- 0.1 (LFP) vs. 0.7 +/- 0.1 (MUA) (TODO FIX)), and the best-fit line offen differed between the two measures, by 30 +/- 10 degrees (TODO). Importantly, 

\subsection*{Co-firing of remote place cells}

\subsection*{Trajectory code synchrony of remote populations}

\section*{Discussion}

\cite{Mizuseki 2012}


% You may title this section "Methods" or "Models". 
% "Models" is not a valid title for PLoS ONE authors. However, PLoS ONE
% authors may use "Analysis" 
\section*{Methods}
All procedures were approved by the Committee on Animal Care at Massachusetts Institute of Technology and followed US National Institutes of Health guidelines. Tetrode arrays were assembled and implanted  according to the procedure in Nguyen et. at. (TODO 2008) and Kloosterman et. al (TODO 2008). We made several modifications to the materials and proceduces to improve our multi-cell sampling.  First, we glued several hundred half-inch pieces of 29 guage and 30 guage hypodermic tubing into rows about 6 mm long, then stacked and glued the rows together to form a honeycomb patterned jig, for organizing the tetrode guide-tubes that would eventually inhabit the microdrive. Second, we developed the ArtE recording system (TODO detailed in Chapter 2) to run in parallel with our usual usual tetrode recording rig. The broader goals of the ArtE project are to enable real-time data analysis and feedback, but in this experiment we used it merely to increase the number of simultaneously recorded tetrodes.

Microdrive arrays were implanted with the center of the grid of tetrodes overlying dorsal CA1 (TODO A/P -4.0, M/L 3.5), spanning 3 mm of hippocampus in the septotemporal dimension and 1.5 mm proximo-distal. In two rats (TODO correct?), tetrodes were lowered into the pyramidal cell layer of CA1 over the course of 2 to 3 weeks and left there for several more weeks of recording.  In two more rats, tetrodes were first lowered into CA1, and later a subset of those was moved further to record simultaneously from field CA3. In each cell layer, we sought to maximize the number of neurons recorded and to minimize within-experiment drift, so closely tracked the shape of sharp wave ripples (which undergo characteristic changes during approach to the cell layer) and later the amplitudes of burgeoning clusters. If either of these factors changed overnight to a degree greater than expected, the tetrode was retracted by 30 - 60 micrometers.

Behavioral training began when nearly all tetrodes exhibited separable spike clusters, and consisted of rewarding rats for simply running back and forth on a curved 3.4 meter linear track, or running continuously counter-clockwise on a 3.4 meter long circular track, with rewards given for every 360 degrees of running for the first 3 laps and for every 270 degrees thereafter. Food deprivation began one or two days prior to the beginning of acquisition, with rats receiving 30 grams of food per day, adjusted up or down depending on the rat's motivation to run and level of comfort (assessed by the amount sleep taken before the running session). The target food-deprived weight was 80\% of free-feeding weight, but we rarely achieved this without disrupting the sleep of the animals, so body weights tended to be 90\% of the free-feeding weight or more, especially after rats learned the simple rules of the task. Additionally, we provided large rewards throughout training (2-5 grams of wetted powdered rat chow per lap), to encourage the long stopping periods during which awake replay can be observed (TODO Foster, 2006). Under these conditions, rats run for about 20 laps or 30 minutes before becoming satiated and ignoring rewards.

Spikes and local field potentials were voltage buffered and recorded against a common white-matter reference, at 32 kHz and 2kHz respectively, and position was tracked at 15 Hz through a pair of alternating LED's mounted on the headstage, as in (TODO) Davidson et. al. (2009). Spikes were clustered manually using the custom program, xclust3 (M.A.W. TODO). Place fields were computed for each neuron in the usual fashion, by partitioning the track into 50 to 100 spatial bins, and dividing the number of spikes occurring with the rat in each spatial bin by the amount of time spent in that spatial bin, in each case only counting events when the rat was moving at least 10 cm/second around the track. Direction of running was also taken into account, allowing us to compute separate tuning curves for the two directions of running, which we label 'outbound' and 'inbound'.

To characterize the phase differences among tetrodes in CA1, a simple spatial traveling wave model was fit to the theta-frequency filtered LFP signals and the theta-filtered multiunit firing rate in turn (TODO figure?), as in Lubenov and Spapas (2011 TODO).

Two complementary techniques were used to assess the relationship between phase offsets between tetrodes and timing offsets in spatial information encoding. First, in CA1-only recordings, a pairwise regression was performed similar to that in Dragoi and Buzsaki (TODO 2006), measuring the dependence of short-timescale peak spike time differences on the distance between the peaks of that pair's place fields. We added a second inedpendent variable to this regression: the anatomical distance between each pair of place cells. The result is a model that predicts the average latency between any pair of cells, given that pair's place fields, that pair's anatomical separation, and the parameters of the traveling wave pattern of phase offsets.

Second, Bayesian stimulus reconstruction (TODO Zhang et. al., 1998) was carried out independently using place cells from thre groups of tetrodes at the most septal end, the middle, or the most temporal end of the 3mm recording grid. Unlike the case for large populations of neurons, reconstructions from smaller anatomical subsets are considerably more noisy and do not reliably yield theta sequences. Session-averaged theta sequences were recovered by aligning the reconstructed position according to a shared theta phase and the rat's position on the track at that time. In both raw and session-averaged reconstruction cases, 2d autocorrelograms were taken to quantify the time-delay and space-delay between pairs of tetrode subgroups.


% Do NOT remove this, even if you are not including acknowledgments
\section*{Acknowledgments}


%\section*{References}
% The bibtex filename
\bibliography{template}

\section*{Figure Legends}
%\begin{figure}[!ht]
%\begin{center}
%%\includegraphics[width=4in]{figure_name.2.eps}
%\end{center}
%\caption{
%{\bf Bold the first sentence.}  Rest of figure 2  caption.  Caption 
%should be left justified, as specified by the options to the caption 
%package.
%}
%\label{Figure_label}
%\end{figure}


\section*{Tables}
%\begin{table}[!ht]
%\caption{
%\bf{Table title}}
%\begin{tabular}{|c|c|c|}
%table information
%\end{tabular}
%\begin{flushleft}Table caption
%\end{flushleft}
%\label{tab:label}
% \end{table}

\end{document}

