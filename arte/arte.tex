% Template for PLoS
% Version 1.0 January 2009
%
% To compile to pdf, run:
% latex plos.template
% bibtex plos.template
% latex plos.template
% latex plos.template
% dvipdf plos.template

\documentclass[10pt]{article}

% amsmath package, useful for mathematical formulas
\usepackage{amsmath}
% amssymb package, useful for mathematical symbols
\usepackage{amssymb}

% graphicx package, useful for including eps and pdf graphics
% include graphics with the command \includegraphics
\usepackage{graphicx}

% cite package, to clean up citations in the main text. Do not remove.
\usepackage{cite}

\usepackage{color} 

% Use doublespacing - comment out for single spacing
%\usepackage{setspace} 
%\doublespacing


% Text layout
\topmargin 0.0cm
\oddsidemargin 0.5cm
\evensidemargin 0.5cm
\textwidth 16cm 
\textheight 21cm

% Bold the 'Figure #' in the caption and separate it with a period
% Captions will be left justified
\usepackage[labelfont=bf,labelsep=period,justification=raggedright]{caption}

% Use the PLoS provided bibtex style
\bibliographystyle{plos2009}

% Remove brackets from numbering in List of References
\makeatletter
\renewcommand{\@biblabel}[1]{\quad#1.}
\makeatother


% Leave date blank
\date{}

\pagestyle{myheadings}
%% ** EDIT HERE **


%% ** EDIT HERE **
%% PLEASE INCLUDE ALL MACROS BELOW

%% END MACROS SECTION

\begin{document}

% Title must be 150 characters or less
\begin{flushleft}
{\Large
\textbf{ArtE - A system for realtime neural ensemble decoding and feedback}
}
% Insert Author names, affiliations and corresponding author email.
\\
Gregory J. Hale$^{1}$, 
Stuart P. Layton$^{2}$, 
Hector Penagos$^{3}$, 
Matthew A. Wilson$^{4,\ast}$
\\
\bf{1} Gregory Hale Brain and Cognitive Sciences, Massachusetts Institute of Technology, Cambridge, Massachusetts, USA
\\
\bf{2} Stuart Layton Brain and Cognitive Sciences, Massachusetts Institute of Technology, Cambridge, Massachusetts, USA
\bf{3} Hector Penagos Brain and Cognitive Sciences, Massachusetts Institute of Technology, Cambridge, Massachusetts, USA
\\
\bf{4} Matthew Wilson Brain and Cognitive Sciences, Massachusetts Institute of Technology, Cambridge, Massachusetts, USA
\\
$\ast$ E-mail: Corresponding imalsogreg@gmail.com
\end{flushleft}

% Please keep the abstract between 250 and 300 words
\section*{Abstract}
Phenomenological descriptions of hippocampal encoding have outpaced our understanding of their underlying mechanisms and ties to learning. Distinct representations are expressed by the same cells in the same circuits, making them impossible to target using traditional molecular methods.  Some disruption specificity can be achieved by leveraging known statistical relationships between information content and the recency of spatial experience, and such experiments have provided the first evidence of a link between sequence replay and learning.  But this method stops short of being able to distinguish among the diverse forms of spatial content known to be expressed in a single recording session.
A method of decoding spatial information content in realtime is needed. To do this, we developed a multi-tetrode recording system focused on streaming representations of the processing stages typically used for offline spatial decoding: spike detection, neural source separation (cluster-cutting), position tracking, tuning curve extraction, and Bayesian stimulus reconstruction.  Our implementation makes critical use of Haskell, a programming language that aides software development by strictly separating a program's logic from its effects on program state, greatly simplifying code and eliminating large classes of common software bugs.  We describe the capabilities and limits of our recording system, its implementation, and routes for contributers to add functionality; and we survey the classes of questions that could benefit from real-time stimulus reconstruction and feedback.

% Please keep the Author Summary between 150 and 200 words
% Use first person. PLoS ONE authors please skip this step. 
% Author Summary not valid for PLoS ONE submissions.   
\section*{Author Summary}
Text

\section*{Introduction}
Text


\section*{Results}

\subsection*{Subsection 1}

\subsection*{Subsection 2}

\section*{Discussion}
\cite{Mizuseki 2012}


% You may title this section "Methods" or "Models". 
% "Models" is not a valid title for PLoS ONE authors. However, PLoS ONE
% authors may use "Analysis" 
\section*{Materials and Methods}

% Do NOT remove this, even if you are not including acknowledgments
\section*{Acknowledgments}


%\section*{References}
% The bibtex filename
\bibliography{template}

\section*{Figure Legends}
%\begin{figure}[!ht]
%\begin{center}
%%\includegraphics[width=4in]{figure_name.2.eps}
%\end{center}
%\caption{
%{\bf Bold the first sentence.}  Rest of figure 2  caption.  Caption 
%should be left justified, as specified by the options to the caption 
%package.
%}
%\label{Figure_label}
%\end{figure}


\section*{Tables}
%\begin{table}[!ht]
%\caption{
%\bf{Table title}}
%\begin{tabular}{|c|c|c|}
%table information
%\end{tabular}
%\begin{flushleft}Table caption
%\end{flushleft}
%\label{tab:label}
% \end{table}

\end{document}

