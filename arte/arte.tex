% Template for PLoS
% Version 1.0 January 2009
%
% To compile to pdf, run:
% latex plos.template
% bibtex plos.template
% latex plos.template
% latex plos.template
% dvipdf plos.template

\documentclass[10pt]{article}

% amsmath package, useful for mathematical formulas
\usepackage{amsmath}
% amssymb package, useful for mathematical symbols
\usepackage{amssymb}

% graphicx package, useful for including eps and pdf graphics
% include graphics with the command \includegraphics
\usepackage{graphicx}

% cite package, to clean up citations in the main text. Do not remove.
\usepackage{cite}

\usepackage{color} 

% Use doublespacing - comment out for single spacing
%\usepackage{setspace} 
%\doublespacing


% Text layout
\topmargin 0.0cm
% sidemargins should be 0.5cm
\oddsidemargin -2.0cm
\evensidemargin 0.0cm
%textwidth should be 16cm
\textwidth 10cm 
\textheight 21cm

% Bold the 'Figure #' in the caption and separate it with a period
% Captions will be left justified
\usepackage[labelfont=bf,labelsep=period,justification=raggedright]{caption}

% Use the PLoS provided bibtex style
\bibliographystyle{plos2009}

% Remove brackets from numbering in List of References
\makeatletter
\renewcommand{\@biblabel}[1]{\quad#1.}
\makeatother


% Leave date blank
\date{}

\pagestyle{myheadings}
%% ** EDIT HERE **

%% PLEASE INCLUDE ALL MACROS BELOW

%% END MACROS SECTION

\begin{document}

% Title must be 150 characters or less
\begin{flushleft}
{\Large
\textbf{Arte - A system for real-time neural ensemble decoding and feedback}
}
% Insert Author names, affiliations and corresponding author email.
\\
Gregory J. Hale$^{1,\ast}$, 
Stuart P. Layton$^{2}$, 
Hector Penagos$^{3}$, 
Alex Chen$^{4}$,
Matthew A. Wilson$^{5}$
\\
\bf{1} Gregory Hale Brain and Cognitive Sciences, Massachusetts Institute of Technology, Cambridge, Massachusetts, USA
\\
\bf{2} Stuart Layton Brain and Cognitive Sciences, Massachusetts Institute of Technology, Cambridge, Massachusetts, USA
\bf{3} Hector Penagos Brain and Cognitive Sciences, Massachusetts Institute of Technology, Cambridge, Massachusetts, USA
\\
\bf{4} Alex Chen, Davis Senior High School, Davis, CA, USA
\\
\bf{5} Matthew Wilson Brain and Cognitive Sciences, Massachusetts Institute of Technology, Cambridge, Massachusetts, USA
\\
$\ast$ E-mail: Corresponding imalsogreg@gmail.com
\end{flushleft}

% Please keep the abstract between 250 and 300 words
\section*{Abstract}
Observational descriptions of hippocampal spatial encoding are outpacing our understanding of their underlying mechanisms and ties to behavior. The traditoinal manipulation techniques can not adequately target the richly choreographed spiking sequences increasingly recognized as an essential feature of spatial encoding. Some disruption specificity can be achieved by leveraging known statistical relationships between information content and the recency of spatial experience, and such experiments have provided the first evidence of a link between sequence replay and learning.  But this method stops short of being able to distinguish among the diverse forms of spatial content known to be expressed in a single recording session.

A method of decoding spatial information content in real-time is needed. To do this, we are developing a multi-tetrode recording system focused on streaming representations of the processing stages typically used for offline spatial decoding: spike detection, neural source separation (cluster-cutting), position tracking, tuning curve extraction, and Bayesian stimulus reconstruction.  Our implementation makes critical use of Haskell, a programming language that aides software development by strictly separating a program's logic from its effects on program state, greatly simplifying code and eliminating large classes of common software bugs.  We describe the capabilities and limits of our recording system, its implementation, and routes for contributers to add functionality; and we survey the classes of questions that could benefit from real-time stimulus reconstruction and feedback.

% Please keep the Author Summary between 150 and 200 words
% Use first person. PLoS ONE authors please skip this step. 
% Author Summary not valid for PLoS ONE submissions.   
\section*{Author Summary}
We report a working, open-source neural recordign system, and a proof-of-concept real-time signal decoding algorithm suitable for interpreting neural activity in rats navigating in mazes. We currently use the system to increase the number of neurons we can simultaneously record from. We intend to integrate the new algorithm in order to feed the information content of the navigating rat's brain back into the experiment, for example, by immediately rewarding the rat for exhibiting brain activity hypothesized to reflect memory formation for specific recently-learned events.

\section*{Introduction}
Temporally compressed spike sequences are an essential feature of hippocampal encoding of space. Each increase in our ability to sample large numbers of cells in freely navigating rats has been accompanied by further support this claim \cite{wilson1993dynamics, miller2008all}.

We are aware of two forms of sequential encoding. The first occurs during active navigation. The majority \cite{thompson1989place} of spiking activity in the hippocampus is due to place cells \cite{okeefe1971the}, which spike only when the rat is within an approximately 1 meter span of the track particular to that place cell (the cell's ``place field''). At any given time, the rat is within the partially overlapping place fields of many place cells. Rather than fire in random order, the spikes are arranged in precise sequences, with spikes from cells with place fields centered just behind the rat first, spikes from place fields centered ahead of the rat last, and a continuum between \cite{skaggs1996theta}. This sequence reflects the sequences of place field centers that the rat would encounter on the track, except it is sped up eight times and repeated once per cycle of the underlying 7-10 Hz ``theta'' oscillation in the local field potential \cite{dragoi2006temporal, foster2007hippocampal}. 

A second form of sequenced spiking occurs while rats are paused on the track, consuming rewards or grooming. At these times, the hippocampus emits irregular, 100-500 ms bursts of activity, with spikes ordered in time according to the spatial ordering of their respective place fields \cite{foster2006reverse, diba2007forward}. These are known as 'sequence replay' events.

In contrast to the large number of studies elaborating on the phenomena of theta sequences and sequence replay \cite{davidson2009hippocampal,gupta2012segmentation,karlsson2009awake,pfeiffer2013hippocampal,cei2014hippocampal}, interventional studies are rare.

Interventional experiments are harder to come by, because the dynamic nature of the activity patterns under study do not lend themselves to genetic or pharmacological targeting. One can not simply mark and purterb some set of cells (even a set defined by activity-dependent genes ~\cite{liu2012optogenetic}), because an overlapping set of cells supports different representations at short timescales ~\cite{thompson1989place,jezek2011theta}.

However, interventional experiments are needed for answering questions about the importance of stimulus-specific activity patters in spatial navigation and memory.




\section*{Results}

\subsection*{Subsection 1}

\subsection*{Subsection 2}

\section*{Discussion}
Text


% You may title this section "Methods" or "Models". 
% "Models" is not a valid title for PLoS ONE authors. However, PLoS ONE
% authors may use "Analysis" 
\section*{Materials and Methods}

% Do NOT remove this, even if you are not including acknowledgments
\section*{Acknowledgments}


%\section*{References}
% The bibtex filename
\bibliography{template}

\section*{Figure Legends}
%\begin{figure}[!ht]
%\begin{center}
%%\includegraphics[width=4in]{figure_name.2.eps}
%\end{center}
%\caption{
%{\bf Bold the first sentence.}  Rest of figure 2  caption.  Caption 
%should be left justified, as specified by the options to the caption 
%package.
%}
%\label{Figure_label}
%\end{figure}


\section*{Tables}
%\begin{table}[!ht]
%\caption{
%\bf{Table title}}
%\begin{tabular}{|c|c|c|}
%table information
%\end{tabular}
%\begin{flushleft}Table caption
%\end{flushleft}
%\label{tab:label}
% \end{table}

\end{document}

